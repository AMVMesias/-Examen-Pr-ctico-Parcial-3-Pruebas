\documentclass[12pt,a4paper]{article}
\usepackage[utf8]{inputenc}
\usepackage[spanish]{babel}
\usepackage{graphicx}
\usepackage{xcolor}
\usepackage{float}
\usepackage{geometry}
\usepackage{fancyhdr}
\usepackage{titlesec}
\usepackage{tcolorbox}
\usepackage{listings}
\usepackage{amssymb}
\usepackage{booktabs}
\usepackage{multirow}
\usepackage{longtable}
\usepackage[T1]{fontenc}
\usepackage[hidelinks]{hyperref}

% Configuración de página
\geometry{margin=2.5cm}
\setlength{\headheight}{14pt}

% Colores personalizados - Tema One Dark Pro
\definecolor{onedarkbg}{RGB}{40, 44, 52}
\definecolor{onedarkfg}{RGB}{171, 178, 191}
\definecolor{onedarkred}{RGB}{224, 108, 117}
\definecolor{onedarkorange}{RGB}{209, 154, 102}
\definecolor{onedarkyellow}{RGB}{229, 192, 123}
\definecolor{onedarkgreen}{RGB}{152, 195, 121}
\definecolor{onedarkcyan}{RGB}{86, 182, 194}
\definecolor{onedarkblue}{RGB}{97, 175, 239}
\definecolor{onedarkpurple}{RGB}{198, 120, 221}
\definecolor{onedarkmagenta}{RGB}{255, 121, 198}
\definecolor{onedarkcomment}{RGB}{92, 99, 112}
\definecolor{onedarkwhite}{RGB}{255, 255, 255}
\definecolor{espeblue}{RGB}{0, 51, 102}
\definecolor{espegold}{RGB}{204, 153, 0}
\definecolor{successgreen}{RGB}{40, 167, 69}
\definecolor{dangerred}{RGB}{220, 53, 69}

% Configuración de listings con tema One Dark Pro
\lstdefinestyle{monokai}{
    backgroundcolor=\color{onedarkbg},
    basicstyle=\ttfamily\small\color{onedarkfg},
    keywordstyle=\color{onedarkpurple}\bfseries,
    keywordstyle=[2]\color{onedarkyellow},
    keywordstyle=[3]\color{onedarkblue},
    keywordstyle=[4]\color{onedarkcyan},
    keywordstyle=[5]\color{onedarkred},
    stringstyle=\color{onedarkgreen},
    commentstyle=\color{onedarkcomment}\itshape,
    numbers=left,
    numberstyle=\tiny\color{onedarkcomment},
    numbersep=10pt,
    breaklines=true,
    showstringspaces=false,
    tabsize=2,
    frame=none,
    xleftmargin=15pt,
    xrightmargin=5pt,
    aboveskip=0pt,
    belowskip=0pt,
    morekeywords={const,let,var,function,return,if,else,throw,new,module,exports,true,false,null,undefined,async,await,describe,test,expect,beforeEach,afterEach,beforeAll,afterAll},
    morekeywords=[2]{require,Error,console,express,jest,supertest,bcrypt,jwt,mongoose},
    morekeywords=[3]{get,post,put,delete,send,listen,log,toBe,toEqual,toHaveBeenCalledWith,mockResolvedValue,mockRejectedValue,toThrow},
    morekeywords=[4]{name,on,push,pull_request,branches,jobs,steps,uses,with,run,main,ubuntu,latest,node,version,actions,checkout,setup},
    morekeywords=[5]{app,res,req,PORT,router,token,user,password,email}
}

\lstset{style=monokai}

% Configuración de tcolorbox para código
\tcbuselibrary{skins, breakable}

\newtcolorbox{codebox}{
    enhanced,
    breakable,
    colback=onedarkbg,
    colframe=onedarkpurple,
    arc=4mm,
    boxrule=1.5pt,
    left=0mm,
    right=2mm,
    top=2mm,
    bottom=2mm,
    shadow={2mm}{-2mm}{0mm}{black!50}
}

% Estilo de títulos
\titleformat{\section}{\Large\bfseries\color{espeblue}}{\thesection}{1em}{}
\titleformat{\subsection}{\large\bfseries\color{espeblue!80}}{\thesubsection}{1em}{}
\titleformat{\subsubsection}{\normalsize\bfseries\color{espeblue!60}}{\thesubsubsection}{1em}{}

% Encabezado y pie de página
\pagestyle{fancy}
\fancyhf{}
\fancyhead[L]{\small\textcolor{espeblue}{Universidad de las Fuerzas Armadas ESPE}}
\fancyhead[R]{\small\textcolor{espeblue}{Pruebas de Software - Práctica Integral}}
\fancyfoot[C]{\thepage}
\renewcommand{\headrulewidth}{0.5pt}

\begin{document}

% Portada
\begin{titlepage}
    \centering
    \vspace*{1cm}
    
    {\Large\textbf{\textcolor{espeblue}{UNIVERSIDAD DE LAS FUERZAS ARMADAS ESPE}}}\\[0.3cm]
    {\large Departamento de Ciencias de la Computación}\\[0.2cm]
    {\large Sede Sangolquí}\\[1.5cm]
    
    \rule{\textwidth}{1.5pt}\\[0.4cm]
    {\Huge\bfseries\textcolor{espeblue}{Práctica Integral}}\\[0.3cm]
    {\LARGE Metodología de Pruebas de Software}\\[0.1cm]
    {\Large Sistema de Gestión de Reservas}\\[0.2cm]
    \rule{\textwidth}{1.5pt}\\[2cm]
    
    \begin{minipage}{0.45\textwidth}
        \begin{flushleft}
            \textbf{Carrera:}\\
            Ingeniería de Software\\[0.5cm]
            \textbf{Asignatura:}\\
            Pruebas de Software\\[0.5cm]
            \textbf{Nivel:}\\
            6to Nivel
        \end{flushleft}
    \end{minipage}
    \hfill
    \begin{minipage}{0.45\textwidth}
        \begin{flushright}
            \textbf{Docente:}\\
            Ing. Enrique Calvopiña E, Mgtr.\\[0.5cm]
            \textbf{NRC:}\\
            27873\\[0.5cm]
            \textbf{Período:}\\
            202555
        \end{flushright}
    \end{minipage}\\[2cm]
    
    \textbf{\large Estudiante:}\\[0.3cm]
    {\Large Mesias Orlando Mariscal Oña}\\[0.2cm]
    {momariscal@espe.edu.ec}\\[2cm]
    
    \vfill
    {\large Sangolquí, Febrero 2026}
\end{titlepage}

% Resumen
\section*{Resumen Ejecutivo}
Este informe documenta la ejecución de un programa integral de pruebas de software aplicando 5 fases metodológicas al Sistema de Gestión de Reservas. Se utilizaron las siguientes herramientas especializadas: ESLint para análisis estático, Jest y Supertest para pruebas unitarias e integración, Postman para pruebas funcionales de API, k6 para pruebas de carga y rendimiento, y GitHub Actions para integración continua.

\textbf{Resultados principales:}
\begin{itemize}
    \item \textbf{23 pruebas automatizadas} implementadas (unitarias + integración)
    \item \textbf{98.61\% de cobertura de código} (superando la meta del 80\%)
    \item \textbf{100\% de pruebas pasadas} exitosamente
    \item \textbf{0 vulnerabilidades críticas} detectadas
    \item \textbf{Pipeline CI/CD} completamente automatizado
\end{itemize}

\textbf{Palabras clave:} Pruebas de Software, Jest, ESLint, k6, CI/CD, API REST, JWT

\tableofcontents
\newpage

%============================================================================
\section{Introducción}
%============================================================================

\subsection{Presentación del Proyecto}
El presente informe documenta la aplicación sistemática de técnicas de pruebas de software al \textbf{Sistema de Gestión de Reservas}, una API REST desarrollada con Node.js, Express y MongoDB que implementa autenticación basada en JWT.

\subsection{Objetivo General}
Aplicar un conjunto integral de técnicas de pruebas de software mediante un enfoque metodológico estructurado, integrando pruebas estáticas, dinámicas, unitarias, de integración, de sistema, de seguridad, de rendimiento y automatizadas en un proyecto web de producción.

\subsection{Objetivos Específicos}
\begin{enumerate}
    \item Ejecutar análisis estático del código fuente para identificar bugs, vulnerabilidades y code smells
    \item Implementar pruebas unitarias con Jest alcanzando cobertura superior al 80\%
    \item Realizar pruebas de integración de la API REST con Supertest
    \item Ejecutar pruebas de carga y rendimiento con k6
    \item Configurar un pipeline de CI/CD con GitHub Actions
    \item Documentar todos los hallazgos y resultados obtenidos
\end{enumerate}

%============================================================================
\section{Marco Teórico}
%============================================================================

\subsection{Pruebas Estáticas}
Las pruebas estáticas analizan el código sin ejecutarlo. Incluyen revisión de código, análisis estático automatizado (linting) y verificación de estándares de codificación.

\subsection{Pruebas Unitarias}
Verifican el comportamiento de unidades individuales de código (funciones, métodos) de forma aislada, utilizando mocks para simular dependencias externas.

\subsection{Pruebas de Integración}
Evalúan la interacción entre múltiples componentes del sistema, verificando que trabajen correctamente en conjunto.

\subsection{Pruebas de Carga}
Miden el rendimiento del sistema bajo diferentes niveles de carga, identificando cuellos de botella y límites de capacidad.

\subsection{Integración Continua (CI/CD)}
Práctica donde los desarrolladores integran código frecuentemente, verificado por construcciones automatizadas para detectar errores tempranamente.

%============================================================================
\section{Materiales y Herramientas}
%============================================================================

\begin{table}[H]
\centering
\begin{tabular}{|l|l|l|}
\hline
\textbf{Herramienta} & \textbf{Versión} & \textbf{Propósito} \\
\hline
Node.js & v20+ & Entorno de ejecución \\
npm & v10+ & Gestor de paquetes \\
ESLint & 9.x & Análisis estático \\
Jest & 30.x & Pruebas unitarias \\
Supertest & 7.x & Pruebas de integración HTTP \\
k6 & Latest & Pruebas de carga \\
MongoDB & 8.x & Base de datos \\
GitHub Actions & - & CI/CD \\
\hline
\end{tabular}
\caption{Herramientas utilizadas en el laboratorio}
\end{table}

%============================================================================
\section{Arquitectura del Sistema}
%============================================================================

El sistema implementa una arquitectura de tres capas con separación clara de responsabilidades:

\begin{itemize}
    \item \textbf{Capa de Rutas:} Define los endpoints de la API REST
    \item \textbf{Capa de Controladores:} Implementa la lógica de negocio
    \item \textbf{Capa de Modelos:} Define los esquemas de datos con Mongoose
    \item \textbf{Middleware:} Maneja la autenticación JWT
\end{itemize}

\subsection{Estructura del Proyecto}
\begin{codebox}\begin{lstlisting}
EvaluacionPractica1-reservas/
├── src/
│   ├── app.js                 # Configuracion Express
│   ├── server.js              # Punto de entrada
│   ├── controllers/
│   │   ├── authController.js  # Logica de autenticacion
│   │   └── reservaController.js
│   ├── middlewares/
│   │   └── auth.js            # Middleware JWT
│   ├── models/
│   │   ├── User.js
│   │   └── Reserva.js
│   └── routes/
│       ├── auth.js
│       └── reserva.js
├── tests/
│   ├── unit/
│   └── integration/
├── k6-tests/
└── .github/workflows/
\end{lstlisting}\end{codebox}

%============================================================================
\section{FASE 1: Análisis Estático}
%============================================================================

\subsection{Configuración de ESLint}
Se configuró ESLint con el plugin de seguridad para detectar vulnerabilidades potenciales:

\begin{codebox}\begin{lstlisting}
// eslint.config.mjs
import js from "@eslint/js";
import security from "eslint-plugin-security";

export default [
  js.configs.recommended,
  {
    plugins: { security },
    rules: {
      "no-unused-vars": "warn",
      "security/detect-object-injection": "warn",
      "security/detect-unsafe-regex": "error",
      "security/detect-eval-with-expression": "error",
      "eqeqeq": "error",
      "no-eval": "error"
    }
  }
];
\end{lstlisting}\end{codebox}

\subsection{Resultados del Análisis}
\begin{table}[H]
\centering
\begin{tabular}{|l|c|c|}
\hline
\textbf{Categoría} & \textbf{Encontrados} & \textbf{Severidad} \\
\hline
Errores & 0 & - \\
Advertencias & 0 & - \\
Vulnerabilidades de Seguridad & 0 & - \\
\hline
\end{tabular}
\caption{Resultados del análisis estático con ESLint}
\end{table}

El código pasó todas las verificaciones de análisis estático sin errores ni advertencias críticas.

%============================================================================
\section{FASE 2: Pruebas Unitarias con Jest}
%============================================================================

\subsection{Configuración de Jest}
\begin{codebox}\begin{lstlisting}
// jest.config.js
module.exports = {
  testEnvironment: 'node',
  coverageDirectory: 'coverage',
  collectCoverageFrom: ['src/**/*.js', '!src/server.js'],
  coverageThreshold: {
    global: {
      branches: 80, functions: 80, lines: 80, statements: 80
    }
  },
  testMatch: ['**/tests/**/*.test.js'],
  verbose: true
};
\end{lstlisting}\end{codebox}

\subsection{Pruebas del Controlador de Autenticación}

\begin{codebox}\begin{lstlisting}
// tests/unit/auth.test.js
describe('AuthController - Pruebas Unitarias', () => {
  describe('register()', () => {
    test('CP-01: Debe registrar un usuario nuevo', async () => {
      mockReq.body = { email: 'test@test.com', password: 'pass123' };
      mockUser.findOne.mockResolvedValue(null);
      mockUser.save.mockResolvedValue({});
      bcrypt.hash.mockResolvedValue('hashedPassword');

      await register(mockReq, mockRes);

      expect(mockRes.status).toHaveBeenCalledWith(201);
      expect(mockRes.json).toHaveBeenCalledWith({ msg: 'Usuario creado' });
    });

    test('CP-02: Debe rechazar usuario duplicado', async () => {
      mockReq.body = { email: 'existing@test.com', password: 'pass123' };
      mockUser.findOne.mockResolvedValue({ email: 'existing@test.com' });

      await register(mockReq, mockRes);

      expect(mockRes.status).toHaveBeenCalledWith(400);
    });
  });
});
\end{lstlisting}\end{codebox}

\subsection{Casos de Prueba Unitarios Implementados}

\begin{table}[H]
\centering
\small
\begin{tabular}{|c|l|l|c|}
\hline
\textbf{ID} & \textbf{Descripción} & \textbf{Resultado Esperado} & \textbf{Estado} \\
\hline
CP-01 & Registrar usuario nuevo & 201 Created & \textcolor{successgreen}{\checkmark} \\
CP-02 & Rechazar usuario duplicado & 400 Error & \textcolor{successgreen}{\checkmark} \\
CP-03 & Manejar error interno registro & 500 Error & \textcolor{successgreen}{\checkmark} \\
CP-04 & Login con credenciales válidas & Token JWT & \textcolor{successgreen}{\checkmark} \\
CP-05 & Login usuario inexistente & 400 Error & \textcolor{successgreen}{\checkmark} \\
CP-06 & Login contraseña incorrecta & 400 Error & \textcolor{successgreen}{\checkmark} \\
CP-07 & Error de BD en login & 500 Error & \textcolor{successgreen}{\checkmark} \\
CP-08 & Crear reserva válida & 201 Created & \textcolor{successgreen}{\checkmark} \\
CP-09 & Error al crear reserva & 500 Error & \textcolor{successgreen}{\checkmark} \\
CP-10 & Asociar reserva a usuario & userId correcto & \textcolor{successgreen}{\checkmark} \\
CP-11 & Acceso sin token & 401 Unauthorized & \textcolor{successgreen}{\checkmark} \\
CP-12 & Acceso con token válido & next() llamado & \textcolor{successgreen}{\checkmark} \\
CP-13 & Token inválido & 400 Error & \textcolor{successgreen}{\checkmark} \\
CP-14 & Token expirado & 400 Error & \textcolor{successgreen}{\checkmark} \\
CP-15 & Token sin Bearer & Procesado & \textcolor{successgreen}{\checkmark} \\
\hline
\end{tabular}
\caption{Casos de prueba unitarios - 15 pruebas pasadas}
\end{table}

%============================================================================
\section{FASE 3: Pruebas de Integración}
%============================================================================

\subsection{Configuración con Supertest y MongoDB Memory Server}

\begin{codebox}\begin{lstlisting}
// tests/integration/api.test.js
const request = require('supertest');
const { MongoMemoryServer } = require('mongodb-memory-server');

beforeAll(async () => {
  mongoServer = await MongoMemoryServer.create();
  process.env.MONGO_URI = mongoServer.getUri();
  app = require('../../src/app');
});

describe('POST /api/auth/register', () => {
  test('CP-INT-01: Registro exitoso', async () => {
    const response = await request(app)
      .post('/api/auth/register')
      .send({ email: 'nuevo@usuario.com', password: 'password123' });

    expect(response.status).toBe(201);
    expect(response.body.msg).toBe('Usuario creado');
  });
});
\end{lstlisting}\end{codebox}

\subsection{Casos de Prueba de Integración}

\begin{table}[H]
\centering
\small
\begin{tabular}{|c|l|l|c|}
\hline
\textbf{ID} & \textbf{Endpoint} & \textbf{Descripción} & \textbf{Estado} \\
\hline
CP-INT-01 & POST /api/auth/register & Registro exitoso & \textcolor{successgreen}{\checkmark} \\
CP-INT-02 & POST /api/auth/register & Rechaza duplicado & \textcolor{successgreen}{\checkmark} \\
CP-INT-03 & POST /api/auth/login & Login exitoso & \textcolor{successgreen}{\checkmark} \\
CP-INT-04 & POST /api/auth/login & Contraseña incorrecta & \textcolor{successgreen}{\checkmark} \\
CP-INT-05 & POST /api/auth/login & Usuario inexistente & \textcolor{successgreen}{\checkmark} \\
CP-INT-06 & POST /api/reservas & Crear con token válido & \textcolor{successgreen}{\checkmark} \\
CP-INT-07 & POST /api/reservas & Rechaza sin token (401) & \textcolor{successgreen}{\checkmark} \\
CP-INT-08 & POST /api/reservas & Rechaza token inválido & \textcolor{successgreen}{\checkmark} \\
\hline
\end{tabular}
\caption{Casos de prueba de integración - 8 pruebas pasadas}
\end{table}

%============================================================================
\section{FASE 4: Pruebas de Carga con k6}
%============================================================================

\subsection{Script de Prueba de Carga}

\begin{codebox}\begin{lstlisting}
// k6-tests/simple-load.js
import http from 'k6/http';
import { check, sleep } from 'k6';

export const options = {
  vus: 50,
  duration: '30s',
  thresholds: {
    http_req_duration: ['p(95)<2000'],
    http_req_failed: ['rate<0.1'],
  },
};

export default function() {
  const user = {
    email: `load_${Date.now()}@test.com`,
    password: 'LoadTest123!'
  };

  // Registro
  const registerRes = http.post(`${BASE_URL}/api/auth/register`, 
    JSON.stringify(user), { headers: { 'Content-Type': 'application/json' }});
  check(registerRes, { 'registro OK': (r) => r.status === 201 });

  // Login y Reserva...
  sleep(0.5);
}
\end{lstlisting}\end{codebox}

\subsection{Resultados de Pruebas de Carga}

\begin{table}[H]
\centering
\begin{tabular}{|l|c|c|}
\hline
\textbf{Métrica} & \textbf{Valor} & \textbf{Umbral} \\
\hline
Usuarios Virtuales (VUs) & 50 & - \\
Requests Totales & 127 & - \\
Tiempo Respuesta (avg) & 1.2s & - \\
Tiempo Respuesta (p95) & 4.33s & <2s \\
Tasa de Errores & 0.00\% & <10\% \\
Throughput & 18.8 req/s & - \\
\hline
\end{tabular}
\caption{Métricas de pruebas de carga con k6}
\end{table}

\textbf{Análisis:} El sistema maneja correctamente la carga con 0\% de errores, aunque el p95 excede el umbral de 2s bajo alta concurrencia.

%============================================================================
\section{FASE 5: Automatización CI/CD}
%============================================================================

\subsection{Pipeline de GitHub Actions}

\begin{codebox}\begin{lstlisting}
# .github/workflows/ci.yml
name: CI Pipeline

on:
  push:
    branches: [ main, develop ]
  pull_request:
    branches: [ main ]

jobs:
  lint:
    runs-on: ubuntu-latest
    steps:
      - uses: actions/checkout@v4
      - uses: actions/setup-node@v4
        with: { node-version: '20' }
      - run: npm ci
      - run: npm run lint

  unit-tests:
    runs-on: ubuntu-latest
    needs: lint
    steps:
      - uses: actions/checkout@v4
      - uses: actions/setup-node@v4
      - run: npm ci
      - run: npm run test:unit

  integration-tests:
    runs-on: ubuntu-latest
    needs: unit-tests
    steps:
      - uses: actions/checkout@v4
      - uses: actions/setup-node@v4
      - run: npm ci
      - run: npm run test:integration
\end{lstlisting}\end{codebox}

\subsection{Jobs del Pipeline}
\begin{enumerate}
    \item \textbf{lint} - Análisis estático con ESLint
    \item \textbf{unit-tests} - Pruebas unitarias con Jest
    \item \textbf{integration-tests} - Pruebas de integración
    \item \textbf{security} - Auditoría de dependencias (npm audit)
    \item \textbf{build} - Verificación de inicio del servidor
    \item \textbf{quality-gate} - Resumen final de calidad
\end{enumerate}

%============================================================================
\section{Resultados Consolidados}
%============================================================================

\subsection{Métricas de Calidad}

\begin{table}[H]
\centering
\begin{tabular}{|l|c|c|c|}
\hline
\textbf{Métrica} & \textbf{Meta} & \textbf{Resultado} & \textbf{Estado} \\
\hline
Cobertura de código & $\geq$80\% & \textbf{98.61\%} & \textcolor{successgreen}{\checkmark} \\
Pruebas pasadas & 100\% & \textbf{100\%} & \textcolor{successgreen}{\checkmark} \\
Errores de linting & 0 & \textbf{0} & \textcolor{successgreen}{\checkmark} \\
Vulnerabilidades críticas & 0 & \textbf{0} & \textcolor{successgreen}{\checkmark} \\
Tasa de errores (carga) & <10\% & \textbf{0\%} & \textcolor{successgreen}{\checkmark} \\
\hline
\end{tabular}
\caption{Métricas de calidad del software}
\end{table}

\subsection{Cobertura por Archivo}

\begin{table}[H]
\centering
\begin{tabular}{|l|c|c|c|c|}
\hline
\textbf{Archivo} & \textbf{Stmts} & \textbf{Branch} & \textbf{Funcs} & \textbf{Lines} \\
\hline
authController.js & 100\% & 100\% & 100\% & 100\% \\
reservaController.js & 100\% & 100\% & 100\% & 100\% \\
auth.js (middleware) & 100\% & 100\% & 100\% & 100\% \\
User.js & 100\% & 100\% & 100\% & 100\% \\
Reserva.js & 100\% & 100\% & 100\% & 100\% \\
auth.js (routes) & 100\% & 100\% & 100\% & 100\% \\
reserva.js (routes) & 100\% & 100\% & 100\% & 100\% \\
app.js & 90.9\% & 100\% & 50\% & 90.9\% \\
\hline
\textbf{TOTAL} & \textbf{98.61\%} & \textbf{100\%} & \textbf{83.33\%} & \textbf{98.52\%} \\
\hline
\end{tabular}
\caption{Cobertura de código por archivo}
\end{table}

\subsection{Resumen de Pruebas}

\begin{table}[H]
\centering
\begin{tabular}{|l|c|c|c|}
\hline
\textbf{Tipo de Prueba} & \textbf{Total} & \textbf{Pasadas} & \textbf{Fallidas} \\
\hline
Pruebas Unitarias & 15 & 15 & 0 \\
Pruebas de Integración & 8 & 8 & 0 \\
\hline
\textbf{TOTAL} & \textbf{23} & \textbf{23} & \textbf{0} \\
\hline
\end{tabular}
\caption{Resumen de ejecución de pruebas}
\end{table}

%============================================================================
\section{Matriz de Trazabilidad}
%============================================================================

\begin{table}[H]
\centering
\small
\begin{tabular}{|l|c|c|c|c|}
\hline
\textbf{Requisito} & \textbf{Unitaria} & \textbf{Integración} & \textbf{Carga} & \textbf{Seguridad} \\
\hline
REQ-01: Crear reserva & CP-08,09 & CP-INT-06,07,08 & \checkmark & \checkmark \\
REQ-02: Autenticación & CP-04,05,06 & CP-INT-03,04,05 & \checkmark & \checkmark \\
REQ-03: Registro usuarios & CP-01,02,03 & CP-INT-01,02 & \checkmark & \checkmark \\
REQ-04: Validación tokens & CP-11,12,13,14 & CP-INT-07,08 & - & \checkmark \\
\hline
\end{tabular}
\caption{Matriz de trazabilidad: Requisitos vs Pruebas}
\end{table}

%============================================================================
\section{Conclusiones}
%============================================================================

\begin{enumerate}
    \item \textbf{Alta Cobertura de Código:} Se logró una cobertura del 98.61\%, superando significativamente la meta del 80\%. Esto garantiza que prácticamente todo el código está respaldado por pruebas automatizadas, reduciendo el riesgo de regresiones.
    
    \item \textbf{Robustez del Sistema de Autenticación:} Las pruebas demuestran que el sistema de autenticación basado en JWT maneja correctamente todos los escenarios: tokens válidos, inválidos, expirados, y solicitudes sin autenticación.
    
    \item \textbf{Estabilidad bajo Carga:} Las pruebas con k6 mostraron que el sistema mantiene una tasa de errores del 0\% bajo carga de 50 usuarios concurrentes, demostrando buena estabilidad operacional.
    
    \item \textbf{Automatización Efectiva:} El pipeline de CI/CD implementado con GitHub Actions asegura que cada cambio en el código pasa por análisis estático, pruebas unitarias y de integración antes de ser integrado, garantizando la calidad continua del software.
\end{enumerate}

%============================================================================
\section{Recomendaciones}
%============================================================================

\begin{enumerate}
    \item \textbf{Optimizar Tiempo de Respuesta:} El p95 de 4.33s bajo carga excede el umbral de 2s. Se recomienda implementar caching, optimizar queries y considerar escalado horizontal.
    
    \item \textbf{Implementar Rate Limiting:} Agregar limitación de peticiones para prevenir ataques de fuerza bruta en endpoints de autenticación.
    
    \item \textbf{Expandir Funcionalidades:} Implementar endpoints GET, PUT y DELETE para reservas, junto con sus respectivas pruebas.
    
    \item \textbf{Pruebas E2E:} Considerar implementar pruebas end-to-end con herramientas como Playwright para flujos completos de usuario.
    
    \item \textbf{Monitoreo en Producción:} Implementar APM (Application Performance Monitoring) para detectar problemas en tiempo real.
\end{enumerate}

%============================================================================
\section*{Repositorio}
%============================================================================
El código fuente completo del proyecto está disponible en:\\
\url{https://github.com/AMVMesias/-Examen-Pr-ctico-Parcial-3-Pruebas}

%============================================================================
\section*{Anexo: Comandos de Ejecución}
%============================================================================

\begin{codebox}\begin{lstlisting}
# Instalar dependencias
npm install

# Ejecutar todas las pruebas con cobertura
npm test

# Ejecutar solo pruebas unitarias
npm run test:unit

# Ejecutar solo pruebas de integracion
npm run test:integration

# Analisis estatico
npm run lint

# Pruebas de carga con k6
k6 run k6-tests/simple-load.js
\end{lstlisting}\end{codebox}

%============================================================================
\begin{thebibliography}{00}
\bibitem{b1} Jest, ``Jest - Delightful JavaScript Testing,'' Jest Documentation, 2026. [Online]. Available: https://jestjs.io/
\bibitem{b2} ESLint, ``ESLint - Find and fix problems in your JavaScript code,'' ESLint Documentation, 2026. [Online]. Available: https://eslint.org/
\bibitem{b3} k6, ``k6 - Modern load testing for developers,'' k6 Documentation, 2026. [Online]. Available: https://k6.io/
\bibitem{b4} GitHub, ``GitHub Actions Documentation,'' GitHub Docs, 2026. [Online]. Available: https://docs.github.com/en/actions
\bibitem{b5} Supertest, ``Supertest - Super-agent driven library for testing HTTP servers,'' npm, 2026. [Online]. Available: https://www.npmjs.com/package/supertest
\end{thebibliography}

\end{document}
